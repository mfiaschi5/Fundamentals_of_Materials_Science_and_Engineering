\section{Viscoelasticità-Lazzeri}
Ora andremo a descrivere un poco meglio le proprietà di un materiale polimerico.
Come accennato in precendenza, i polimeri sono una classe di materiali che permettono estensioni molto grandi in cui non si può più ragionare in termini di allungramento puramente elastico. La \textbf{rubber-like elasticity} è caratterizzata curve stress-strain in cui si arrivano a valori di strain fino a 10x il valore iniziale, caratterizzato da non linearità estesa. Per ottenere queste proprietà è necessario un certo grado di cross-linking, in caso opposto la le catene polimeriche scorrono l'una sulle altre creando così delle deformazioni non elastiche.
Come accennato in precedenza, queste proprietà dei polimeri sono dovute principalmente a motivi entropici più che a motivi termodinamici, in quanto l'ordine aumenta quando il materiale e sottoposto ad una forza esterna.
Ci sono diversi modelli che spiegano questa iper-elasticità:
\begin{itemize}
    \item Modello della freely jointed chain: considera il polimero non come una struttura da angoli definiti in cui son presenti atomi ma come un insieme di elementi rigidi di lunghezza fissata e assume che tutte le configurazioni abbiano la stessa energia. Queste catene sono collegate da un crosslink che lega le catene.
\end{itemize}