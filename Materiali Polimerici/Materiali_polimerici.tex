\newpage
\section{Materiali Polimerici}
I polimeri sono materiali formati da molecole a catena molto lunga in cui si ha la ripetizione di un'unità più piccola detta \textbf{\textit{monomero}}. Esistono vari tipi di polimeri in base alle loro proprietà, una possibile classificazione è:
\begin{itemize}
    \item \textbf{\textit{Termoplastici}}: i materiali termoplastici possono essere fusi e diventano più duttili all'aumentare della temperatura; sono gli unici che possono presentare un certo grado di cristallinità.
    \item \textbf{\textit{Termoindurenti}}: Non possono essere fusi poichè comporta la rottura dei legami e si ha un aumento del modulo elastico all'aumentare della temperatura; sono sempre amorfi. 
    \item \textbf{\textit{Elastomeri}}: permettono di allungarsi di diverse volte il loro valore iniziale (anche x10); sono sempre amorfi.
\end{itemize}
Le proprietà di un polimero vanno ricercate direttamente nella loro composizione.
In generale ci possono essere più monomeri diversi: omopolimero se il tipo di monomero è solo 1, copolimero random se sono disposti casualmente, alternato o a blocchi o graft (con il back-bone di un tipo e le ramificazioni di un altro). Il tipo di polimero dipende molto dal metodo in cui è stato prodotto. I polimeri statistici, random e alternati di solito hanno proprietà intermedie ai singoli omopolimeri. Questa cosa permette di modulare le proprietà in base alla composizione.
Il polimero a blocchi o ramificato mostra proprietà tipiche di ogni monomero inoltre questi possono mostrare nuove proprietà a causa del legame chimico tra le catene non uguali. la presenza di \textbf{\textit{cross-links}} è estremamente importante ed incide fortemente le proprietà del polimero. Perchè si formino \textbf{\textit{cross-links}} è necessario la presenza di gruppi funzionali che permettano la creazione di nuovi legami. Un'altra cosa che determina fortemente le proprietà del polimero è il peso molecolare. In generale sperato un certo valore di soglia non si ha una sostanziale differenza nelle proprietà anche se si ha un aumento molto grande della viscosità ed un conseguente aumento nella difficoltà di lavorazione. Eisstono più modi di calolare la massa dei polimeri, è possibile fare una semplice media aritmetica o mmedie pesate. Il rapporto tra la media pesata e la media aritmedica si chiama \textbf{\textit{polidispesione o heterogenity index (PDI)}} e ci dice quanto è omogenea la composizione all'interno del nostro polimero. \textcolor{green}{immagine viscosità vs peso molecolare e proprietà}. Il \textbf{\textit{grado di polimerizzazione}} è calcolato come il numero di unità presenti calcolati a partire dalla massa media. Nel caso di copolimeri la massa usata è una media pesata sulle frazioni molari. Alla'mentare della massa si osserva un'aumeto della durezza e si passa da cere o solidi malleabili a solidi duri. I polimeri possono presentare un certo grado di cristallinità in base a diversi fattori. 
\subsection{Termoplastici}
La proprietà principale dei termoplastci è la possibilità di poter essere rimodellati diverse volte. Non tutti i materiali possono essere cristallizzati ed a più si ottengono strutture semi-cristalline. Le fasi cristallina posseggono un $T_m$ al di sopra della quale la fase cristallina diventa liquida; quelli amorfi invece possseggono la gass transition temperature in cui da un solido vetroso (duro) si ottiene uno stato rubbery (duttile). Le T$_m$ sono relativamente basse a causa della debolezza dei legami intercatena; T$_m$ e T$_g$ aumentano all'aumentare della rigidezza e delle forze intermolecolari.
\textcolor{green}{tabella}
\subsection{Elastomeri e termoindurenti}
La caratteristica principale di queste due classi è la presenza di cros-links, i primi hanno una bassa densità di cross-links, la seconda molto elevata. L'elasticità di questi materiali si basa su  entropy recovery mechanism. La csa da tenere in considerazioe è che una volta formati i cross-links questi materiali non possono essere ulteriormente lavorati.
\subsection{polimerizzazione}
\begin{itemize}
    \item chain growth
    \item step growth
\end{itemize}
Possono procedere per addizione o condensazione: il polietilene si ottiene per polimerizzazione radicalica a catena (è un addizionr), questo itpo di tecnica viene usata per monomeri insaturi. In questo tipo di polimerizzazione la catena che sta crescendo ha una reattività maggiore del singolo monomero (difatti è un radicale) ma indipendente dalla dimensione della catena stessa. Alti livelli di polimerizzazione sono ottenuti anche a basse coneversioni in quanto la crecita è indipendente, l'aumento del rate di conversione spesso si riflette sul numero di polimeri ottenuti più che sul grado di polimerizzazione in quanto una volta che la catena è morta non reagisce èiù.
Illustriamo ora gli step di crscita:
\begin{itemize}
    \item Inizializzazione ad opera di luce/calore/radicale. La polimerizzazione inizia a causa della presenza di radicali, una specie molto reattiva che da luogo ad una reazione a catena. La formazione del radicale può avvenire attraverso da degli iniziatori che formano radicali se sottoposti a radiazione o a calore.
    \item Propagazione: il radicale si "trasferisce" quando il monomero si lega alla catena. Nel caso del cloruro di vinile (per fare il PVC) si hanno due possibilità di propagazione (testa-testa o testa coda), la probabilità rispetta diversi fattori come l'energia di attivazione dei due processi e la stabilità termodinamica dei due radicali; nel caso del PVC la forma più stabile è quella che presenta il radicale nel cabronio legato al cloro. 
    \item terminazione : può avvenire per diversi motivi: due catene che si stanno propagando si combinano, si può avere disproporzione tra le due catene o si può avere trasferimento di catena.
    \end{itemize}
\subsection{carrelata di polimeri}
    \begin{itemize}
        \item PE: bassa densià, bassa resistenza ma alto strain (600\%). tre tipi: HDPE, LLDPE, LDPE. Viene usato per imballaggi e rivestimenti e come isolante elettrico. pf 110
        \item PVC: amprfp, fragile, alta resistenza chimica, rogodo ma poco resiliente, difficilemnte lavorabile. Può essere unito con diversi additivi: lubrificanti per migliorare la lavorabilità, fillers per abbatterne i costi, pigmenti per colorarlo. stabilizzanti termici (composti organimetallici). Usato er giubbotti waterproof, tetti delle macchine, film e containers.
        \item Poliproilene pbasso costo, buona durezza superficiale e discreta resistenza all'midità, calore e agli agenti chimici. è usato per fare container, oggetti domesitici, 
        \item Polistirene fragile e non flessibile e trasparente. La resistenza può essere aumentato con l'aggiunta di polibutadiene. Facilmente processabile ed a basso costo. Scarsa reistenza agli agenti atmosferiche. è usato anche per gli imballaggi sottoforma di polistirolo.
        \item PAN (poliacrilonitrile) fibra per vestiti, buona resistenza maeccanica e stabilista chimica
        \item SAN poslistirebe e acrilonitrile.  EIgido duro e trasparente. Pannelli e lentied anche tazze o cose per uso domenstico o vetri di sicurezza.
        \item PTFE 
        \item Polimetilacrilato (PMMA): plexiglass 
        \item ABS (acrilonitrile/butadiene/polistirene): buona resistenza meccanica e resistenza chimica, le proprietà cambiano in base alla composizione
    \end{itemize}
    \subsection{step-growth polimerization}
    Richiede che il monomero abbia almeno due gruppi funzionali.
    La crescita non è spontanea e la reazione avviene lentamente e monomeri e polimeri presentano reattività pressoche invariata. Il grado di polimerizzazione continua a crescere all'aumentare della conversione. Possono coinvolgere due specie diverse  una specie dello stesso tipo. avviene per reazioni successive. Le reazioni sono spesso policondensazioni. Un esempio è il PET che è fatto a partire da acido ftalico e glicole etilenico. I poiesteri sono fatti a partire da biacidi e alcoli, le poliammidi (nylon) a partire da ammine e acidi carbossilici. Policarbonati, polisulfoni. Se invece abbiamo la formazione di un legame tra monomeri senza prodotti scondari si parla di poliaddizione linear come nel caso delle resine epossidiche. L network step policondansation consiste in due step:
    formazione di prepolimeri a bassa massa e successivo raggiungimento di alti livelli di crosslinks dopo aver sottoposto il campione a calore. Tipici monomeri sono l'urea ed il fenolo o la formaldeide. Alcuni esemoio sono le resine fenoliche. La anche l'addizione può formare networs come il caso del DGEBA che da luogo a strutture ramificate a partire da epossidi e diammine. gelazione: all'aumentare della conversine si ha un aumento non lineare della viscosità, Superato un certo punto (quando le dimensioni delle molecole sono comparabili con quelle del sistema) il nostro materiale non si comporta più come un liquido e diventa immobilizzato, le molecole possono ancora muoversi ma attraversp la diffusione; questo impedisce la conversione completa. Due modi di prevedere Carothers e flory. Il primo sovrastime $p_c=\frac{
    2
    }{f_{av}}$ assume che la gelazione avviene a lunghezza infinita, cosa falsa; il secondo sottostima ($p_ap_b)_c=\frac{1}{f-1}$non tiene in conto effetti intramolecolare che possono creare loop e spostare a conversioni più alte il punto di gelazione.
    \subsection{Cristallinità nei polimeri}
    Alcuni polimeri possono cristallizzare raggiungendo un certo grado di cristallinità (mao completa). Il graod massimo dipende da tanti fattori:
    \begin{itemize}
        \item La velocità di raffreddamento, la tatticità, la massa molare, la presenza di additivi come agenti nucleanti o la presenza di catene laterali. La crisytallizzazione può essere evitata raffreddando molto velocemente il polimero, la cristallizzazione può essere indotta dopo con l'annealing ad una temperatura tra $T_g$ e $T_m$. Livelli pi alti di cristallizzazione possono essere ottenuti se si raffreddano i polimeri a partire da soluzioni diluite piuttosto che dal fuso a causa di assenza di fattori sterici. La struttura tipica è la lammellae. Esistono diversi difetti (immagine). La presenza di gruppi polari da direzionalità al tutto, siu ha un legame selettivo tra particolari gruppi funzionali.  meccanismo adjacent e random re-entry (come si formano le lamelle. Sferuliti: sono simili ai grani dei metalli.
        Condizioni da soddisfare per la cristallizazione
    \end{itemize}
    temperatura di fsuione: non è un valore preciso ma un range, dipende dalla storia del campione, dipende dal rate di riscaldamento. Inoltre dipende dalla struttura chimica, dalla massa molare, dalle ramificazioni e dalla copolimerizzazione. Come trovare la glass transition a partire dal volume specifico. i tipi di polimeri che danno $T_g$ e i vari statii (mettere immagine di slide 87. vedere slide 90, molto importante, praticamente dice che i materiali cross linked dopo la T$_g$ presentano un moldulo di Young crescente al crescere della temperatura. Caoutchouc: polisopropilenne (mettere la formula) c'è sia il cis edd il trans, quest'ultimo tende a cristallizzare invece. L'aggiunta di zolfo e il riscaldamento genera la gomma, quella vulcanizzata è molto più resistente. GOmme sintetiche: polistrirene+polibutadiene (SBR per le gomme) , oliacrilonitrile+butadiene per appicazioni in cui è richiesta maggior resistenza chimica e meccanica. Policloropìprene (1-clorobutadiene o neoprene) buona resistenza chimica e meccanica, è usato per ricoprimenti e cavi. Gomma di silicone (struttura) Isolante.
    \subsection{Proprietà meccaniche}
    For some polymers, strong stretching with a stress 
minimum and break at a much higher stress is 
observed. 
This phenomenon is known as strain hardening and stress induced crystallization. It is caused 
by orientation and alignment of polymer chains parallel to the load which increases both 
strength and stiffness of the plastic in stretch direction.
Comporatemento degli elastomeri: allungamento pazzesco a bassi strain ed improvviso aumento dello stress prima della rottura, occhio sempre a strain rate, T densità di cross links, peso molecolare, composizione e microstruttura (esempio di polimero con microstrutture molto diverse).
Effetto di T sulla rottura, si osserva una transizione brittle to fragile (spesso coincide con T$_g$) a volte è minore come il PC. Comportamento viscoelastico: materiali entanglati leggermente (come gli elastomeri) daranno grandi deformazioni elastiche  prima di rompersi, quelle non crosslinkate come i termoplastici daranno.
\textcolor{red}{il comportamento elastico non è necessariamente lineare? perhce all'aumenatre di T i termoindiìurenti induriscono? dopo il T glass i termoindurenti rammolliscono? infine anche gli elastomeri si comportano così?}compostamento viscoelastico. Il comportamento viscoelastico si osserva al di sopra di T$_G$. Nel regime viscoelastico si ha una dipendenza dal rate e dal tempo in cui è stato sottoposto il carico. I materiali viscoelastici sono materiali che in parte hanno un comportamento elastico (legge di hooke) ed in parte tipico dei materiali liquidi (legge di Newton), la natura della risposta del polimero cambi al variare di T, a bassa t il regime elastico domina, ad alta T l'incontrario.
legge di newton:
\begin{equation}
    \tau=\eta\frac{d \epsilon}{dt}
\end{equation}
Esistono diversi modelli tutti fatti da molle e pozzetti. Creep function (??). Ogni modello ha i suoi vantaggi e svantaggi (vedere Lazzeri). Boltzmann superposition principle :  linera viscoelasticity ha come assunzione questo principioo.  La somma delle deformazioni è la somma algebrica dello strain a causa dei step di carico. ALla fine è possibile scrivere le cose come somma (che si approssima ad intergale che si traduce in una convoluzione quando si fa la trasformata.
In generale la trasformata del modulo elastico è complesso:
\begin{equation}
    E(\omega)=E'(\omega)+E"(\omega)
\end{equation}
La parte complessa è associata all'energia dissipata per ciclo mentre la parte reale è l'energia immagazzinaa, il rapporto di parte reale e parte complessa e definita come \textbf{\textit{damping factor}}:
\begin{equation}
    \tan(\delta(\omega))=\frac{E'(\omega)}{E"(\omega)}
\end{equation}
In generale è possibile plottare E in funzione della frequenza, i picchi possono essere considerati come effetti di damping ad una frequenza caratteristica di transizione o rilassamento del processo associato al particolare moto dei polimeri.
\textcolor{green}{aggiungere i grafici} Dai grafici si vede che se la frequenza del carico è lenta si ottiene lo stesso effetto di alzare la temperatura. Vicevera ad alta T si comporta come un solido vetroso.
gnificant damping occurs at Tg when the applied test frequency matches the 
natural frequency for main-chain rotation. At higher frequencies, there will be 
insufficient time for chain uncoiling to occur and the material will be relatively 
stiff. On the other hand, at lower frequencies, the chains will have more time 
to move and the polymer will appear to be soft and rubbery. n general, in any viscoelastic system, the transition process associated to a 
particular type of molecular motion can be observed not only by a change in 
the timing of the stimuli, but also by a change in the diffusivity of the system, 
i.e. by a change in the temperature at which the response is tested. \textcolor{green}{mettere grafici dei presenti in slide 115}. It is thought that there is a general equivalence between time and temperature. 
For instance, a polymer that displays rubbery characteristics under a given set 
of testing conditions can be induced to show glassy behaviour by either 
reducing the temperature or increasing the testing rate or frequency. 
In generale due curve calcolate a T' e T e a frequenze differenti $\omega$ e $\omega'$ mostrano la sressa compilance ma a T differenti, esiste solamente un shift factor $\alpha$ che dipende esclusivamente da T. la funzione di $\alpha_T$ è fittata con la WLF euation che ha due parametri diversi in base ai vari polimeri. In uno spettro di G in funzione della temperatura (o della frequenza)si possono osservare diversi picchi, questi picchi sono dovuti a diverse transizioni dette primarie e secondarie. Ogni transizione è associata ad un particolare moto molecolare.queste analisi si effettuano con il DMTA dynamic mechanical thermal analysis. Mettere un paio di immagini spiegando bene la compilance e cosa c'è in ogni grafico.  The average molecular mass of chains primarily affects the mechanical resistance since 
a minimal degree of polymerization is mandatory in order to obtain a mechanically 
stable solid with appreciable stiffness and resistance. However, beyond a critical level 
of molecular mass there is no more increment of both mechanical stiffness and 
resistance with the length of the chain.
 As the crystallinity degree grows, for a fixed quantity of material, both tensile strength, 
elastic modulus and density are increased. Anche la struttura del cristallo è importante per la determinare le proprietà meccaniche. La presenza di gruppi laterali aumenta il modulo di Young poichè impedisce il movimento delle catene polimeriche. Anche la presenza di gruppi molto polari aumentano molto la resistenza meccanica. Anche gruppi laterali molto ingombranti stericamente. L'uso di fibre di vetro comportanoun un aumento delle proprietà della resistenza meccanica. é possibile definire il modulo di creep ($mc=\sigma/\epsilon(t)$ e lo stress relaxation (stessa cosa ma ciò che varia nel tempo è il carico.
\subsection{Frattura nei polimeri}
può essere duttile o fragile o una cosa intermedia. a T inferiore della T$_G$la frattuura è per lo più di tipo fragile, viceversa duttile. La frattura fragile è spesso accompagnata dalla formazione di zone cave e di fibre in cui i polimeri sono altamente allineati. (quesdte zone sono chiamate craze.la formazione del craze e diversa da quella di una frattura in quanto la craze può ancora supportare carico finchè nonsi arrriva ad una elongazione tale da rrivare alla rottura.La frattura duttile avviene quando è possibile lo slipping e quindi la deformazione plastica del materiale. nel processo anche qui si formano delle fibre a causa della forza.
\subsection{Ciclo industriale}
Si parte dai materiali raw (petrolio) per poi ottenere i monomeri. Si fa la polimerizzazione o direttamente su stampi o si ottengono prima pellets, polveri o soluzioni liquide che comunque richiederanno ulteriori processi. ci sonodiversi modi per poliaddizione:
\begin{itemize}
    \item in \textbf{bulk}: qui la concentrazione di monomero è massima ed il processo relativamente  più semplice dando alti rates di polimerizzazione e bune conversioni, inoltre è spesso difficiel rimuovere efficacemente il calore (anche a causa dell'elevata viscosità) dal sistema e la viscossità aumenta moto rapidamente. Si usa poco inizializzatore per ridurre il rate e si usano basse conversioni al costo di recuperare il monomero non reagito.
    \item In soluzione sia il monomero che il polimero che l'inizializzatore sono sciolti. Il solvente deve ridurre complessivamente la viscosità  e permettere ko scambio di calore. La concentrazione di monomero e minore inoltre bisogna rimuovere il solvente (per evaprazione) o riuscendo a far precipitare il polimero. filtrazioni e centrifugazioni fanno il resto.
    \item in sospensione serve per evitare probei causati dl trasferimento di calore ma con a reazione che si comporta come fosse nel bulk. IL medium deve essere inerte nei confronti delle altre specie involte. Spesso si usa l'acqua limitando la temperatura di lavoro a 70-90 gradi. 
    \item in emulsione. Qui The reaction components differ from those used in 
suspension polymerization only in that the initiator 
must not be soluble in monomer but soluble only in the aqueous dispersion medium; questa cosa modifiva il meccanismo e la cinetica di polimerizzazione e anche il prodottocreando un particolato di polimero in acqua conosciuto come Latex.
    \end{itemize}
    Only for thermosets:
    \begin{itemize}
        \item estrusione
        \item injection moluding (valido anche per i ternoindurenti, la cura si fa nello stampo.)
        \item bubble film blowing
        \item Thermoforming
        \item blow moulding
        \item compression moulding
    \end{itemize}
La frattura all'interno dei polimeri dipende molto dal grado di cristallinità. 